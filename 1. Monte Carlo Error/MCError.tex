\documentclass[border=5mm, convert, usenames, dvipsnames,beamer]{standalone}
\usetheme{Madrid}
\usecolortheme{default}
%Information to be included in the title page:
\title{Sample title}
\author{Anonymous}
\institute{Overleaf}
\date{2021}

\usepackage[absolute,overlay]{textpos}

\defbeamertemplate*{frametitle}{}[1][]
{
    \begin{textblock*}{12cm}(1cm,0.75cm)
    {\color{purple} \fontsize{20}{43.2} \selectfont \insertframetitle}
    \end{textblock*}
    \begin{textblock*}{12cm}(1cm,2.5cm)
    {\color{purple} \fontsize{20}{24} \selectfont \insertframesubtitle}
    \end{textblock*}
}



\setbeamertemplate{footline}[frame number]
\usepackage{ragged2e}

\justifying
\usepackage{lmodern}
\usepackage{ImageMagick}
\usepackage[utf8] {inputenc}
\usefonttheme[onlymath]{serif}
\usepackage[english] {label}
\usepackage{amssymb}
\usepackage{amsmath}
\usepackage{bm}
\usepackage{bbm}
\usepackage[round] {natbib}
\usepackage{color}     
\usepackage{changepage}
\usepackage[export]{adjustbox}
\usepackage{graphicx}
\usepackage{enumitem}
\usepackage{minted}
\usepackage{mathtools}
\usepackage{listings}
\usepackage[svgnames]{xcolor}


\lstset{language=R,
    stringstyle=\color{DarkGreen},
    otherkeywords={0,1,2,3,4,5,6,7,8,9},
    morekeywords={TRUE,FALSE},
    deletekeywords={data,frame,length,as,character},
    keywordstyle=\color{blue},
    commentstyle=\color{teal},
}
\newcommand{\a}{\item[\boldmath$ \mathclap{\color{red}(i)}$ ]}
\newcommand{\b}{\item[\boldmath$ \mathclap{\color{red}(ii)}$ ]}
\newcommand{\c}{\item[\boldmath$ \mathclap{\color{red}(iii)}$ ]}
\newcommand{\d}{\item[\boldmath$ \mathclap{\color{red}(iv)}$ ]}


\makeatletter
\setbeamertemplate{frametitle}[default]{}
\makeatother
\usepackage{booktabs}


\begin{document}

\begin{frame}[ fragile]{}
\frametitle{Monte Carlo error in simulation-based \\ statistical analyses: introduction}

\vspace{45}
\textbf{After Elizabeth Koehler, Elizabeth Brown and Sebastien J.-P. A. Haneuse (2009)}

\vspace{20}
\noindent
$(i)$ \hspace{7} Draw attention on the importance of reporting Monte Carlo (MC) error. 

\vspace{8}
\noindent
 $(ii)$\hspace{9} Provide simple and practical tools for estimating MC error.

\vspace{8}
\noindent
 $(iii)$ \hspace{3} Provide means for determining the number of replications required to achieve a prespecified level of accuracy. \\


\end{frame}





\begin{frame}[ fragile]{}

\frametitle{Experiment 1:  
estimators of operating \\ characteristics of the log-odds ratio (1/5)}

\vspace{50}
\noindent
Assumed generating process:  \ \ $Logit(Y=1 \mid X) = \beta_{0} + \beta_{1}x

\vspace{10}
\noindent
MC error of an estimator $\hat{\phi}$: \ \  $MCE(\hat{\phi}) = \sqrt{var(\hat{\phi})}$

\vspace{10}
\noindent
\textbf{Setting 1}: 

\vspace{4}
\noindent
 $(i)$  Quantification of association between $Y$ and $X$, two binary r.v.

\vspace{4}
\noindent
 $(ii)$ Assessment of some operating characteristics of the Maximum Likelihood Estimator (MLE) of the slope (i.e. log-
odds ratio)

\vspace{20}
\noindent
N (sample size) = 100

\vspace{4}
\noindent
R (number of replicates) = 100, 500, 1,000, 2,500, 5,000, 10,000

\vspace{4}
\noindent
M (number of simulations) = 1,000

\end{frame}






\begin{frame}[ fragile]{}

\frametitle{Experiment 1: three operating \\ characteristics (2/5)}


\vspace{65}
\noindent
$\hat{\phi}_{R}^{b}$, an estimate of the \textbf{percent bias} for the MLE of $\beta_{X}$, defined as:

\vspace{-4}

$$
\hat{\phi}_{R}^{b} = \frac{1}{R} \sum_{r=1}^{R} \frac{\hat{\beta}_{X}^{r} - \beta_{X} }{\beta_{X}} * 100
$$

\noindent
$\hat{\phi}_{R}^{c}$, an estimate of the\textbf{ coverage rate of the $\mathbf{95}$ \% Confidence interval}, defined as:

\vspace{-4}

$$
\hat{\phi}_{R}^{c} = \sum_{r=1}^{R} \ \mathbbm{1} \   \bigg[  \hat{\beta}_{X}^{r} - 1.96 \ se( \hat{\beta}_{X}^{r}})  \geq   \beta_{X}     \leq \hat{\beta}_{X}^{r} + 1.96 \ se( \hat{\beta}_{X}^{r})  \bigg]
$$

\noindent
$\hat{\phi}_{R}^{p}$, an estimate of the\textbf{ power} to detect an association, defined as:

\vspace{-4}

$$
\hat{\phi}_{R}^{p} = \sum_{r=1}^{R} \ \mathbbm{1} \   \bigg[   \left|  \frac{ \hat{\beta}_{X}^{r}     }{  se( \hat{\beta}_{X}^{r})} \right|  < \hat{\beta}_{X}^{r}   \bigg]
$$


\end{frame}



\begin{frame}[ fragile]{}

\frametitle{Experiment 1: MC estimates of percent \\ bias (3/5)}

\vspace{15mm}
\noindent
\begin{figure}[h!]
\begin{center}
\includegraphics[width=10cm]{Picture1}
\caption{}
\end{center}
\end{figure}




\end{frame}





\begin{frame}[ fragile]{}

\frametitle{Experiment 1: R code to generate the \\ graph (4/5)}

\vspace{50}
\noindent


\par
\tiny
\begin{lstlisting}[language=R]
beta11=numeric(10000)
...
beta61=numeric(10000)

beta0= -1  ; betax = log(2)    # known parameter for logistic regression
Px1=0.3 ; Px0=0.7   # P(X=1) and P(X=0)
levels=c(1,0) ; N1=100 ; X=c(rep(1,30),rep(0,70))  # fixed

set.seed(1)
for(i in 1:10000)
{  z1 = beta0 + betax *X      # linear combination with a bias
  Py1 = 1/(1+exp(-z1))   # pass through an inv-logit function
  y1 <- rbinom(N1,1,Py1)
  beta11[i] <- glm(y1 ~ X, family="binomial"(link='logit'))$coeff[2]  }
...
set.seed(6)
...
# figure 1
maxden=10000 ; den=1:maxden
hplotR61=cumsum(round((round(beta11,2) - (betax*rep(1, maxden))) / 
               (betax*rep(1, maxden)),2)*100) / den
...
hplotR65=cumsum(round((round(beta51,2) - (betax*rep(1, maxden))) /
               (betax*rep(1, maxden)),2)*100) / den
plot(hplotR61,type="l", xlim=c(1,maxden), ylim=c(-5,5), main="MC estimates of bias ..."
     ,xlab="Number of replicates R", ylab="Bias", col = 'red3')
lines(hplotR62)
...
lines(hplotR65)
\end{lstlisting}

\par
\end{frame}



\begin{frame}[ fragile]{}

\frametitle{Experiment 1: results (5/5)}


\vspace{15mm}
\noindent
\begin{figure}[h!]
\begin{center}
\includegraphics[width=7.5cm]{Picture2}
\caption{}
\end{center}
\end{figure}





\end{frame}












\begin{frame}[ fragile]{}

\frametitle{Experiment 2:  quantification of MC \\ error
 (1/4)}

\vspace{60}
\noindent
\textbf{Setting 2}: 

\vspace{10}
\noindent
By the strong Law of Large Numbers, we have that  $\hat{\phi}_{R} \rightarrow E \big[  \phi(X) \big]$

\vspace{5}
\noindent
By the Central Limit Theorem, we know that $\sqrt{R} (\hat{\phi}_{R} - \phi) \rightarrow N \big(0, \sigma_{\phi}^{2} \big)$

\vspace{15}
\noindent
We consider the two following measures:


\begin{flalign}
  (i) \ \ \hat{MCEclt} = \frac{ \hat{\sigma}_{\phi}}{R}   = \frac{1}{R}  \sqrt{   \sum_{r=1}^{R}  \big( \phi(X) - \hat{\phi}_{R}  \big)^{2}   }   &&
\end{flalign}


\begin{flalign}
 (ii) \ \ \hat{MCEboot} =  \frac{1}{B}  \sqrt{   \sum_{b=1}^{B}  \big( \hat{\phi}_{R}( \mathbf{X}_{b}^{*}) - \overline{\hat{\phi}_{R}(\mathbf{X}_{b}^{*}})  \big)^{2}   }  &&
\end{flalign}




\end{frame}






\begin{frame}[ fragile]{}

\frametitle{Experiment 2: results (2/4)}



\vspace{15mm}
\noindent
\begin{figure}[h!]
\begin{center}
\includegraphics[width=10.5cm]{Picture3}
\caption{}
\end{center}
\end{figure}




\end{frame}





\begin{frame}[ fragile]{}

\frametitle{Experiment 2: more results (3/4)}


\vspace{15mm}
\noindent
\begin{figure}[h!]
\begin{center}
\includegraphics[width=10cm]{Picture4}
\caption{}
\end{center}
\end{figure}





\end{frame}





\begin{frame}[ fragile]{}

\frametitle{Experiment 2: partial R code (4/4)}

\vspace{30}
\noindent


\par
\tiny
\begin{lstlisting}[language=R]
betaA2=numeric(100)
...
betaF2=numeric(10000)
betamean100=numeric(1000)
...
betamean10000=numeric(1000)

# R=100
set.seed(3)
for(j in 1:1000)  {
  for(i in 1:100)
  { beta0= -1     # known parameter for logistic regression
    betax = log(2)   # [1] 0.6931472  parameter for logistic regression
    levels=c(1,0)
    X=c(rep(1,30),rep(0,70))  # fixed
    z1 = beta0 + (betax *X)      # linear combination 
    Py1 = 1/(1+exp(-z1))   # inv-logit function
    y1 <- as.numeric(rbinom(100,1,Py1))
    betaA2[i] <- glm(y1 ~ X, family="binomial"(link=logit))$coeff[2]      }
  betamean100[j] = mean(betaA2)       }
...
calcPB <- function(data, index, truth) (mean(data[index]) - truth) / truth * 100

calcSE <- function(data, index) sd(data[index])

mceBoot <- function(data, B, type="", truth=NULL) {...}

## for  M=500,000
# R=100
set.seed(3)
mceBoot(betamean100_2, B=100,type="PB", truth=betax)
# [1] 0.009481879
\end{lstlisting}

\par
\end{frame}




\begin{frame}[ fragile]{}

\frametitle{Some remarks }

\vspace{}
\noindent
$(i)$ \hspace{9} The magnitude of the MC error seems to be linear in $1 / \sqrt{R}$, (see plot next slide).

\vspace{10}
\noindent
$(ii)$ \hspace{9}  As $1 / \sqrt{R}$ tends to $0$, MC error tends to $0$, (see plot next slide).

\vspace{10}
\noindent
$(iii)$ \hspace{8} An idea: maybe we could use a constrained linear regression to estimate R to achieve an acceptable level of MC error.

\end{frame}






\begin{frame}[ fragile]{}

\frametitle{Magnitude of the MC error as a \\ function of R}


\vspace{15mm}
\noindent
\begin{figure}[h!]
\begin{center}
\includegraphics[width=11cm]{Picture5}
\caption{}
\end{center}
\end{figure}





\end{frame}



\begin{frame}[ fragile]{}

\frametitle{Conclusions}

\vspace{15}
\noindent
$(i)$ \hspace{9} The MC error can be more substantial than traditionally though.

\vspace{10}
\noindent
$(ii)$ \hspace{9}  Even after $500,000$ simulations, there is residual uncertainty.

\vspace{10}
\noindent
$(iii)$  \hspace{8} The magnitude of MC error depends on various factors, i.e. parameters, operating characteristics, variability in the data, etc.


\vspace{10}
\noindent
$(iv)$  \hspace{9} The MC error can be drastically reduced by increasing R, the number of replicates.



\end{frame}




\begin{frame}[ fragile]{}

\frametitle{References}

\vspace{35}
Elizabeth KOEHLER, Elizabeth BROWN, and Sebastien J.-P. A. HANEUSE, 2009. On the Assessment of Monte Carlo Error in Simulation-Based Statistical Analyses.

\vspace{15}

Christian P. ROBERT, George CASELLA, 2005. Monte Carlo Statistical Methods.

\vspace{15}

Maria L. RIZZO, 2008. Statistical computing with R.

\vspace{15}
F. H. C. MARRIOTT, 1979. Barnard's Monte Carlo Tests: How Many Simulations? 

\vspace{15}
\textbf{Full R code of this reproduced study available upon request}.




\end{frame}


\end{document}
